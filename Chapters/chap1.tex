%----------------------------------------------------------------------------------------
%	CHAPTER 1
%----------------------------------------------------------------------------------------
\cleardoublepage
\chapterimage{chap_head.png} % Chapter heading image

\chapter{Homopolar Motor}
A motor is, elementarily, a device which converts electrical energy into
mechanical energy and kids know it as something which moves or rotates.
But one may question as to how exactly do these motors produce the rotational
force. This rotational force is due to the interaction between an electric current
and a magnetic field oriented in a specific direction. This basic principle can
be seen in action and appreciated better with a simple demonstration of a
homopolar motor. The homopolar motor, first created in 1821 by Michael
Faraday, is the most fundamental example of a motor, and is really fun to
experiment with.

\section{What all you need}
\begin{itemize}
    \item Duracell AA Battery(1.5V)
    \item Tapered Screw
    \item Insulated Copper Wire
    \item Neodymium Magnet (diameter - 12mm)
    \item A piece of paper (optional)
\end{itemize}

\section{Get ready to build the world’s simplest motor}
There is no better way to learn how a motor works than by building your very
own motor! Here is what you have to do:
\begin{itemize}
    \item[Step 1:] Attach the head of the screw to the magnet as shown in Fig
    \item[Step 2:] Connect the pointed side of the screw to the negative terminal of the battery as shown in Fig 
    \item[Step 3:] Use the insulated copper wire to connect the positive terminal of the battery to the side of the 
        magnet. Be careful that the wire should touch the side of the magnet at right angles and not at its bottom. 
\end{itemize}

\section{Understanding the principle behind}
When the screw is connected to the magnet, it gets magnetized. As soon as the wire is connected from one end of the 
battery to the side of the magnet (and not the bottom), the circuit gets completed. The rotation is produced because 
the direction of current and that of the magnetic field is perpendicular to each other. In this case, a force (known 
as Lorentz Force) is exerted in a direction perpendicular to BOTH of them, causing the spinning motion. Fig 2.5 specifies 
the directions of the Current, magnetic field and force. The direction is given by Fleming’s Left Hand Rule (shown in 
Fig 2.6); which says that if the thumb, forefinger and middle finger of the left hand are stretched to be perpendicular 
to each other, and if the fore finger represents the direction of magnetic field, the middle finger represents the 
direction of current, then the thumb represents the direction of force.

The following mathematical equation represents the relationship:
\begin{equation*}
    F = I(\overrightarrow{L} \times \overrightarrow{B})
\end{equation*}
The stronger the magnet, the faster the wire will rotate. Neodymium magnets are the strongest in the world and therefore, 
while using an AA battery to make a homopolar motor, an ideal magnet to use is a 12mm diameter and a 6mm thick neodymium magnet.

\section{Thinking Beyond}
Now, you know how the simplest motor works and it is, in fact, true for all of the DC motors that you can think of. From 
the industrial fans, household devices like the food processor, food mixers, electric watches to the largest instances in 
automotive and ship propulsion, this principle is followed. Generally, an electric motor would contain a coil of wire that 
can create an electromagnetic field aligned with the centre of the coil, when electricity flows through it. This wire is 
referred to as an electromagnet. The magnetic field may be created using permanent magnets as well. When, a current is sent 
through this loop of wire, the electromagnet experiences a Lorentz force on it at right angles to the direction of the 
magnetic field and the current flowing. And this force causes the loop of wire to generate mechanical energy and rotate.

Why is this motor called homopolar motor? The homopolar motor gets its name from the fact that the direction of the electrical
current and the magnetic field never reverses, unlike an AC motor, or a DC motor containing a commutator.

What more can you do? One may add a paper fan as shown in Fig 2.7 to this homopolar motor to see the motor actually rotating, 
even from a far distance. Just by adding a piece of paper or some plastic blades between the screw head and the magnet, you 
can see that as the screw rotates, the blades/paper rotates!

You can even bend a wire to form any shape, for example, a wired Ballerina! The shape of the wire should be such that allows 
it to balance properly, otherwise it might fall off the battery when it begins to spin.