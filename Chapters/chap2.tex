%----------------------------------------------------------------------------------------
%	CHAPTER 2
%----------------------------------------------------------------------------------------
\cleardoublepage
\chapterimage{chap_head.png} % Chapter heading image

\chapter{Faraday Generator}
Science is taught to us through textbooks. But not only can we observe
various physical laws in real, we can also apply them to make cool stuff
that demonstrates them. For example, conversion of one form of energy to
another is a fascinating event to see. The glowing LED faraday generator is a
perfect demonstration of Faraday’s law. The fundamental principles behind
the generation of electricity were based on Michael Faraday’s discoveries in
the nineteenth century. This DIY kinetic generator converts your mechanical
power (just a number of shakes) to electrical power, based on his principle.

\section{What all you need}
To make your own glowing LED Faraday Generator, you will need the
following materials:
\begin{itemize}
    \item Thread spool $\times$ 1
    \item Enameled copper wire: 36-SWG $\times$ 50m
    \item Red LEDs $\times$ 2
    \item Neodymium Magnet (diameter- 7 mm) $\times$ 1
    \item Matchsticks $\times$ 2
    \item Needle/Compass $\times$ 1
    \item Sand Paper
    \item Watchmaker’s File $\times$ 1
    \item Insulation Tape
    \item Soldering Equipment*
\end{itemize}

\section{Understanding the principle behind}
Before we apply the science, we must know the science. Michael Faraday, a
master experimentalist of the nineteenth century, is best known for his invention of the phenomena of electromagnetic induction and that of the electric
generator and motor. This was, in fact, his most revered contribution to two
of the most coveted and bewildering subjects of science and innovation at
that time, Electricity and Magnetism. His work formed a strong basis for the
merger of these two, and its significance is paramount. The invention of an
electric generator and motor finds application in almost every appliance today.
From washing machines to fans, to food mixers and electric chargers, all use
Faraday’s theory of electromagnetic induction. The glowing LED faraday
generator works on the Faraday’s law of electromagnetic induction which
states that:

\textbf{\emph{“The induced electromotive force (or voltage) in any closed circuit is
equal to the negative of the time rate of change of the magnetic flux
enclosed by the circuit.”}}

Here, your closed circuit is going to be the 50 m of copper wire wound
around the thread spool and the magnetic flux is changed by shaking the
thread spool. This changes the magnetic field due to the up-down motion of
the magnet. The change of magnetic field as seen by the wire, generates an
electromagnetic force(EMF) that is proportional to the number of turns of
wire, as well as the rate at which the magnetic field is changed.

\section{Get Ready to Shake and Generate}
To make your own LED glowing faraday generator, all you need to do is to
follow a set of few simple steps:

\begin{itemize}
    \item[Step 1:] Start by creating two marks on the thread spool, with the help of pen
        or a marker, such that the distance between them is equal to the length
        of the magnet.
    \item[Step 2:] Now, with the help of a needle, pierce the thread spool at these marks
        as shown in Fig 3.2 such that you have holes on both the sides of the
        thread spool. Be careful, it would hurt if the needle pierces your finger
        instead of the thread spool.
    \item[Step 3:] Using a watchmaker’s file, file those holes and inside of the thread
        spool for smooth movement of the magnet inside the spool. Also, make
        sure that the holes are big enough to put a matchstick in them. Once
        the filing is done, put two matchsticks in these holes as shown in Fig
        3.3.
    \item[Step 4:] Now, we start winding the copper wire. Leave 6 to 8 cm of wire, wind
        5-6 turns on one of the matchsticks and then start winding on the thread
        spool. By doing this, we fix one end of the wire. Be careful that you
        wind the copper wire between the matchsticks only. Have patience, it
        will take some time.
    \item[Step 5:] Once the winding is done, put tape around the windings and then
        remove the matchsticks. Be careful not to stick ends of the wire inside
        the tape. 
    \item[Step 6:] Since we are using ‘enameled’ copper wire, we need to remove the
        enamel at the ends of the wire to make the connections. Sand the ends
        using a sandpaper very slowly and lightly, or otherwise, the wire might
        break.
    \item[Step 7:] Now is the time for soldering. Solder two LEDs at the ends of the
        wire in opposite polarities (You can identify the polarity of an LED by
        seeing the length of its legs, the longer one is positive and the shorter
        one is negative). 
\end{itemize}
\textbf{NOTE:} 

The reason for opposite polarity of LEDs is Lenz’s law which states
that:

\textbf{\emph{“The direction of current induced in a conductor by a changing
magnetic field due to Faraday’s law of induction will be such that
it will create a magnetic field that opposes the change that produced it.”}}

Hence, if a clockwise current is induced due to the upward motion of
the magnet, then an anti-clockwise current will be induced due to the
downward motion of the magnet. Thus, if one LED glows when the
magnet goes up, the other will glow when the magnet goes down.
\begin{itemize}[resume]
    \item[Step 8:] Now stick the LEDs on the thread spool using tape. Stick them 180
        degrees apart so that you can see both of them glowing separately. And
        you are done! Your glowing LED Faraday Generator is ready. Insert
        your magnet inside the spool and as you move the thread spool updown, you can see the LEDs glowing.    
\end{itemize}

\section{Thinking Beyond}
This Faraday Generator, when shaken vigorously, can be used to generate
sufficient energy to charge up capacitors. This energy stored in capacitors
can be utilized in powering low power circuits. How about your TV Remote?
You can get rid of those batteries with a Kinetic Remote or an Electronic Dice
which gives a random number when you shake it!