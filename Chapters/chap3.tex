%----------------------------------------------------------------------------------------
%	CHAPTER 3
%----------------------------------------------------------------------------------------
\cleardoublepage
\chapterimage{chap_head.png} % Chapter heading image

\chapter{DC Motor}
A DC motor is a device which converts electrical energy into mechanical
energy. One may find DC motors in many portable home appliances like
fans, food mixers, in automobiles and in numerous industrial equipment. The
rotational force is due to the interaction between an electric current and a
magnetic field, both of which are oriented in specific different directions.
This manual will help you make a very impressive motor of your own!

\section{What all you need}
\begin{itemize}
    \item Zero PCB
    \item Hook up wires
    \item Insulated copper wires (single strand)
    \item Enameled copper wire(21 SWG)
    \item Neodymium magnets (diameter -12mm)
    \item Wire cutter
    \item Sand paper
    \item Double sided tape
    \item Duracell AA battery(1.5V)
    \item Paper Cutter
    \item Soldering Equipment
\end{itemize}

\section{Get Ready to make your own DC Motor!}
\begin{enumerate}
    \item[Step 1:] Creating the coil
        \begin{enumerate}
            \item Take the enameled copper wire (21 SWG) and cut a length of
                around 1 feet from it with the help of wire cutter as shown in Fig
                4.2.
            \item Take the AA battery and wrap the wire over the battery leaving approximately 2 inches of wire from each of the ends as shown
            in Fig 4.3.
            \item The coil should be tightly wound because a balanced coil will
            rotate more swiftly.
            \item Wrap the ends of the coil 2 or 3 times through the coil to make
            the coil maintain its shape.
            \item The trailing ends should be diametrically opposite to each other,
            for the symmetrical shape and for the balance of the motor.
        \end{enumerate} 
    \item[Step 2:] Removing the insulation from the ends of the coil 
        \item[] 
            The next step is a bit difficult and, so, has to be performed with care. The
            insulation of the ends of the coils has to be removed, to unleash the
            metal underneath it, to make it conducting. Take one end of the coil
            and start removing the enamel, using a paper cutter, leaving a shiny
            new layer of copper wire. The enamel from the end of the coil has to be
            removed from only one half of the perimeter of the wire, say, the upper
            half. Similarly, the insulation from the other end has to be removed
            from only the upper half of the perimeter. This process is done so that
            the force exerted by the magnet to the coil should be unidirectional,
            and the motion is sustained.The enamel should be removed properly to
            ensure a good electrical connection between the wire and coil. After
            this, you will have a coil that is ready to be the armature of your motor,
            as shown in Fig 4.6.
    \item[Step 3:] Creating posts for the motor coil to rotate
        \item[] The posts will form a stand for your motor, aiding its rotation.
        \begin{enumerate}
            \item To start building a stand for the motor, you need a Zero Board
            and two hookup wires.
            \item Measure the length of your coil end to end leaving 1 inch on
            either end and mark it on the Zero Board.
            \item Take a hookup wire of nearly 2 inches length and make it straight
            using a tweezer.
            \item Then make a small U-shape at one end of it through a plier/tweezer.
            \item Put it through the mark on the Zero Board from downwards side.
            Press the smaller end of the U-shaped post in the zero board, so
            that the wire stands straight. Repeat the process for the other wire
            as well.Now, you have two wire posts (each of lengths 2 inches)
            standing straight on the Zero Board. Solder the Hookup Wires in
            place.
            \item Using the plier, make a loop at the top end of each of the wires.
            Ensure that the loops are similar to both the posts.Now take your
            coil and try to suspend it with the help of these loops made on the
            posts.
            \item If it suspends in a perfectly horizontal fashion, then proceed.
            Otherwise, adjust the posts and the height of the loops to make it
            suspend horizontally.
        \end{enumerate}
    \item[Step 4:] Making the electrical connections
        \item[] For the electric current to flow through your coil, you need to connect
            the battery to the motor
        \begin{enumerate}
            \item Take two single strand copper wires, preferably 15 cm in length.
            \item Strip off its insulation up to half inches from either side, for both
            the wires.
            \item Take one wire and wrap one of its ends at the bottom end of a
            post, thus establishing a connection. Do the same with the other
            wire as well. Make sure that the connection should be strong
            enough to hold.            
        \end{enumerate}
    \item[Step 5:] Attaching the magnet
        \item[] Now it is time for assembling the most important part of this motor, the
            magnet.
        \begin{enumerate}
            \item Take the magnet and place it at the centre of the two posts and
            mark its place.
            \item Now cut a piece of double-sided tape and paste it at the mark
            you just made for the magnet. Take off the other covering of the
            tape at put the magnet there. Apply some pressure so that it holds
            good. Take care that the magnet should be around 2-3 cm below
            where the coil would be, and perfectly under it.        
        \end{enumerate}
    \item[Step 6:] Testing the motor and making the necessary twitches
        \begin{enumerate}
            \item Suspend the copper coil on the wire posts. Make sure that the
            exposed ends should make contact with the hookup wires.
            \item Take up the battery and fix one of one of the copper wires to the
            end of the wire through a tape, preferably, to fix a connection.
            Take the other end and complete the electrical circuit by touching
            it to the other end of the battery.
            \item The copper coil may start rotating on its own, but may also,
            sometimes, require a small push to start rotating. If the coil is not
            spinning after this, you may need to sand the ends of the coil or
            you may need to make the coil more balanced to make it rotate.
            Keep trying till it doesn’t start rotating!
        \end{enumerate} 
\end{enumerate}

\section{Understanding the principle behind}
A motor is an electrical machine which converts electrical energy into mechanical energy. The working principle behind in a motor is that, whenever
a current carrying conductor is placed in a magnetic field, it experiences
a force, according to the Fleming’s Left-Hand Rule (discussed on page 7).
When the armature coils are connected to a DC voltage supply, some current
starts flowing in the conductor. The magnetic field may be provided by electromagnets or by using permanent magnets. In this case, current carrying
armature conductors experience a force due to the magnetic field provided by
a neodymium magnet present near the base of the coil.